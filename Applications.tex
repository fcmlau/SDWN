\section{Applications in {\sdn}}
\label{App}

\subsection{Overview}

Traditional applications cannot achieve complicated goals and efficiency due 
to the limited processing power and memory space of sensors.

In {\sdn}, applications for wireless sensor networks are inspired by 
greater potential with the UAV based SDN controller. The central controller
helps sensors execute complex calculations such as AI model training, as well 
as store global information. Besides, UAVs have flexible features and can deploy 
tasks to sensors by one-hop communication directly. Thus it enables the sensor network
to achieve much more intelligent applications.

In {\sdn}, applications can be found for a variety of purposes, including routing, network diagnosis,
AI energy prediction, AI node selection and multi-task. We design all these applications and provide 
easy-to-use interfaces to users as in Table \ref{API}.



\subsection{Routing}
\label{subsectionrouting}

\subsubsection{Motivation}

Routing is a fundamental application of SDN system. There are lots of studies on SDN in wired network, but only few researches on wireless sensor networks since the implementation of SDN routing in WSN have the following challenges. First, the wireless communication is unstable, the poor reliability of connection between controller and sensor nodes makes SDN system performance worse, especially the controller connects with nodes through multi-hop wireless approach. 
Second, the sensor network is energy sensitive, the deployment of SDN downward stream energy cost needs to be ignored, compare with the normal network data communication, and the frequency of routing update should low. 
Last but not least, achieving the robustness of SDN routing in WSN is challenging because the controller and nodes may be disconnected. When some nodes' radio fails, it may cause the entire network connection crashing.

To address the above challenges, we use a mobile controller which communicates with nodes via one-hop to improve the reliability of nodes and controller connection. Meanwhile, we design a cluster-based SDN routing protocol and a local repair mechanism to achieve both energy efficiency and robustness.




% Different from the internet protocol (IP) networks which adopt
% routing with shortest path,
% traditional WSNs use the RPL (Routing Protocol for LLN) protocol,
% focusing on routing with less energy consumption.

% Low-Power and Lossy network

% The remarkable of SND is its 

% is feasible using access control lists (ACLs) and routing protocols. Users can provide 

% determine the path of network.

% As we discuss before, sensors are limited to their processing capability and storage space.
% Thus we simplified flow table and 

% Traditional wireless sensor network RPL \cite{Winter2012}.

% Open Shortest Path First (OSPF) is a routing 
% protocol for Internet Protocol (IP) networks.

% In our , we implement OSPF in the 

% The flow table of the 

% one round

% cluster header


\subsubsection{Design}
In WSNs, cluster routing method is a kind of protocol, which nodes choose a header periodically, the other node around cluster header send it's data to cluster header, and the cluster header forwards it to another header, until the root node receives the data. 
Though cluster method will reduce the energy hole problems, the routing protocol between is running on sensor nodes locally, because sensor nodes only have local information of network and limited computing ability, it can not achieve global optimal routing, and it is difficult to modify protocol according users requirements.

We improve cluster method by using our SDN controller to update the routing protocol between cluster headers. 
At first, a UAV flies around the sensor field and get topology data from nodes, 
and then calculates the cluster header schedule to decide which node should be the cluster header at a specific time phase 
to balance energy consumption between nodes, and calculate the routing result between current header's topology through
Open Shortest Path First (OSPF) protocol base on gathered topology information, In WSN, every node only need send its data to root node, so the OSPF destination internet protocol (IP) has be set as root IP. Through OSPF, UAV get the routing table between cluster header. When UAV fly near the nodes, it send routing table to nodes, including the node role and next hop IP in every time phase.

Considering the economic and energy limitation, UAV can not keep fly above the sensing field. It only fly to sensing field to gather topology after node deployed and update nodes routing on demand. So after the routing has been deployed, the system routing repair mechanism should make sure some nodes failure don't make the network routing crashed.


By designing local routing repair mechanism, our system achieves robustness to radio failure. When a node in cluster member failure, it didn't influences the whole routing between the routing we deployed routing. On other case, when the cluster header failure, if the network don't have local repair mechanism, the whole routing between cluster headers will crash and make the network partial. So we design the local repair mechanism through local header switching. 

\begin{figure}[htbp]
	\centering
	\includegraphics[width=.85\columnwidth]{Figure/routing-flow}
	\vspace{-0.1in}
	\caption{Routing local reparation}
	\label{routing-flow}
	\vspace{-0.1in}
\end{figure}

As shown in Figure~\ref{routing-flow}, after the node get routing table from
UAV, it sets its next hop IP following received table, and also uses its
neighbor's RSSI information to find the nearest node as its substitute node,
then node broadcast its next hop IP and substitute node IP. After the substitute
node received the broadcast packet, it save the source's next hop IP and keep
monitoring the source status. The substitute node will became new header when it
detected the current header failed.

\begin{lstlisting}[
	language={[ANSI]C},
	label=topology-update,caption={An example of deploy routing algorithm},
	keywordstyle=\color{blue!70},
	showstringspaces=false,
	commentstyle=\color{red!50!green!80!blue!70},
	frame=single,captionpos=t,
	rulesepcolor=\color{red!20!green!20!blue!20},
	basicstyle=\ttfamily]
topology = get_topology();
//calculate route table for each node
//based on topology
for(node=0; node<nude_num; node++){
   node.routingtable =
      calculate_routable(topology);
}
//set route table for each node
for(node=0; node<nude_num; node++){
   fly_to(node);
        set_route(node.routingtable);
}

\end{lstlisting}



We implement our software defined routing and local reparation in {\sdn} and demonstrate the code in List \ref{topology-update}, the routing in our system achieve both energy efficiency and robustness.


\subsection{Network Diagnosis}

\subsubsection{Motivation}

Sensor nodes are provisioned with low-capacity batteries and will run out of energy in the end. 
Besides, uncertain environmental factors will lead to the failure of communications.
Upon these two factors, there occurs network faults such as failure nodes and lossy links from time to time.

Network fault diagnosis is then designed to help network administrators monitor the network 
operational status and maintain a sensor network system. The key idea of the existing works
on fault diagnosis is to collect running information from nodes and deduce root causes of network
exceptions. The running information is collected by a mechanism of probing.

We implement the network diagnosis process as an application in our {\sdn} system.
Since in {\sdn} we have our mobile UAV as a controller, a more flexible way is to infer the suspicious 
nodes of the network fault in traditional way first and then set the UAV to check out the fault sources. 
 
\subsubsection{Design}

We first introduce a state-of-the-art algorithm named DID 
\cite{gong2015directional}, which is a directional diagnosis approach.
We utilize the node tracing module and the tracing collection module of DID
to infer the suspicious nodes in our network diagnosis application, 
and then set our UAV controller to confirm the network elements being faulty.

The node tracing module is conducted in each sensor node. 
Every time a packet arrives at a node, it counts
the source of the packet . The tracing collection module
is set in the UAV controller. When network exception is detected, 
it gathers the tracing information in the tracing module of the relevant nodes
and infers a suspicious node set. Next we set the UAV to fly through all these nodes 
and send beacons to them to diagnose the failed nodes. 
Then UAV collects the neighbor lists of every node in the suspicious node set.
With the collected neighbor lists, the UAV controller reconstructs the topical topology and 
compares it with the default topology to find the failed links.

Compared to DID, the diagnosis application in {\sdn} releases the complicated inference
computations and can achieve accurate diagnosis since the UAV can fly to the sensors 
to confirm the network faults. {\sdn} can realize the four types of fault sources the same as DID:  
\begin{itemize}
\item	Node failure. This network failure is caused by the node itself.
\item	Link failure. This network failure is caused by the communication links 
between nodes, mainly relating to traffic flow in networks.
\item	Temporary failure. This  network failure is caused by complex interior or exterior 
interferences and quick self-recovery
\item	Multiple failures. This  network failure is caused by multiple failures above.
\end{itemize}

\subsection{AI Energy Prediction}

\subsubsection{Motivation}
Traditional wireless sensor networks (WSN)  generally carry out data 
collection by multi-hop transmission,
and due to limited computation capability and storage capacity, sensors are incapable
to perform machine learning methods for data mining. 
Hence, intelligent self energy consumption optimization for a sensor is hard to achieve.

The sensor next to the data center always has the most frequent routing tasks, which leads to
the heaviest workload and the shortest battery life. Worse more, after the
first sensor runs out of energy, a new sensor will be selected to take the place 
of it with more workload. As a result, its lifetime becomes shorter and the whole 
sensor network will soon run out of energy. We call this energy exhausted problem.

An ideal solution to the energy exhausted problem is to find a self-adjusted routing schedule which 
makes adjustments according to the energy status of each sensor. 
Actually, this is a kind of dynamic programming problem and
the optimization target is to achieve the longest lifetime of the whole network under
complex constrains. 

In {\sdn}, the UAV controller stores the global energy information 
and can perform large-scale computations, laying 
the foundation for the intelligent energy prediction.
The AI prediction in turn can contribute to the whole system  network diagnosis application
This kind of cooperation and mutual assistance makes {\sdn} an effective ecosystem.



\subsubsection{Design}

\begin{figure*}[htbp]
	\centering
	\includegraphics[width=5in]{Figure/SDWN-AIprediction}
	\caption{AI prediction}
	\label{prediction}
\end{figure*}

In {\sdn},  we implement an energy prediction application 
to forecast lifetime of sensors. The architecture of this application
can be seen in Fig. \ref{prediction}.


We utilize the popular machine learning model, 
Multi-layer Perceptron (MLP) \cite{Harvey1988An},
to conduct the energy prediction process. 
The AI-based prediction architecture is  showed in Fig. \ref{prediction} .
After the information collected by the sensor is transmitted to database, it is used as
the input of the incremental learning model, which outputs the
life expectancy of each sensor. 

A sample input of energy information contains at least four components: sensors'
duty cycle, packet-sending time, data receiving time and other energy
consumption. Data collection module records these information with time stamp
as well as routing information. We use the database to maintain our experimental data,
and use the sql query to manage training data. In this way, we can easily
get each sensor's routing flows and energy consumption at any time as input. A forgetting factor
is used to avoid overfitting. Then the model will output the prediction of a energy consumption list in time
sequences. The sensor which is most likely to run out of energy will rank in the first place of the list. 
Combined with our diagnosis and routing application, we can make use of our prediction result to prolong the network
lifetime.





\subsection{AI Node Selection}

\subsubsection{Motivation}

It is inevitable that there will be a part of redundant sensors when deploying a 
practical wireless sensor network. These redundant nodes have overlaps of
observation regions, and what makes the matter worse is that redundant nodes
may cause great communication interference. Therefore it is significant to select 
proper sensors to avoid data redundancy and save the sensor network energy consumption.

In {\sdn}, we provide the node selection application to users. The SDN controller executes the 
selecting algorithm and send the control instructions to activate the selected nodes.

\subsubsection{Design} Our {\sdn} system provide two main node selecting methods: 
greedy selection algorithm and SRSSS algorithm. This application will be extended to more elegant 
algorithms in our future work. 

\textbf{Greedy selection algorithm.} We first provide a simple method to select 
the redundant nodes by a greedy selection algorithm, as described in Alg. \ref{Greedy}. 
The key idea is to select nodes as less as possible to coverage the whole area
 based on the location and sensing range. 
 
We implement the greedy selection algorithm in {\sdn}. The evaluations in section \ref{Eva} 
wshow it greatly saves the sensors' energy and thus prolongs the network lifetime.

\begin{algorithm}
\caption{Greedy Selection Algorithm}
\label{Greedy}
\begin{algorithmic}[1]
\STATE Input: Sensor set $N$, Selected set $M$, Target area $\Omega$, Covering area $\Phi$;
\STATE Initialize : $M = \emptyset$, $\Phi = \emptyset$
\WHILE {$M \neq N$}
    \IF{$\Phi = \Omega $}
        \STATE break; $\backslash$$\backslash$ Selected set has been found
    \ENDIF
    \IF{$\forall n_i \in (N-M) : range(n_i) \subset \Phi$}
    	 \STATE break;$\backslash$$\backslash$ Cannot cover the target area;
    \ENDIF
    \STATE Find $n_i : argmax(\Phi \cap range(n))$, $n_i \in (N-M)$;
    \STATE $\Phi = \Phi \cup {n_i}$
\ENDWHILE
\STATE Output: $M$;
\end{algorithmic}
\end{algorithm}

\textbf{Spatially regularized streaming sensor selection (SRSSS).} 
To realize more intelligent and effective sensor selection, we introduce 
a state-of-the-art AI algorithm named spatially regularized streaming 
sensor selection (SRSSS) \cite{li2016spatially}.

Different from the greedy selection algorithm, SRSSS is a multi-variate 
interpolation framework and focuses on selecting a subset
of sensors in a streaming scenario to minimize collected information redundancy.  

Traditional wireless sensor network is not suitable to implement an AI selection approach
due to the limited computational capability of sensors. Some work use the database to collect data
and make decisions by multi-hop communications. However, in this way the network will use up a great deal of energy.
In our {\sdn}, the UAV fly through the nodes to pick up the collected data and executes the computations.
Then it sends the control instructions to the nodes by one-hop communication and greatly saves the network energy.

The aim of SRSSS is to optimize its objective function which is an equation given
certain constraints of collected information, location and energy consumption.
The objective function is formulated as:

\begin{equation}
\label{OF}
\begin{aligned}
& (W_{k+1},z_{k+1})  \\
& = arg \min_{W,z} \sum_{i=1}^k \mu^{k-i}\lVert X_k^iD_zW(I-D_z)-X_k^i(I-D_z)\rVert^2_2 \\
& + \alpha\sum_{i,j=1}^n\lVert y_i-y_j\rVert_2\lvert W_{i,j} \rvert - \beta\sum_{i,j=1}^n\lVert y_i-y_j\rVert_2 z_iz_j \\
& + \lambda\lVert W \rVert^2_F \\
&s.t. z = [z_1,...,z_n] \in {\{0,1\}}^n, c^Tz \leq P
\end{aligned}
\end{equation}

The first term in (\ref{OF}) is to minimize the prediction error of the collected data
and the following two terms incorporate spatial information. The last term constrains
the complexity of the learned matrix $W$. The energy constraint is controlled by 
the inequality in (\ref{OF}). Because of the limitation of length,  we leave out all the details. 
The meaning of the parameters and the mechanism of 
SRSSS can be seen in \cite{li2016spatially}. With the AI sensor selection process, 
{\sdn} becomes a smarter and adaptive sensor system. 






\subsection{Multi-task}
\label{subsectionmultitask}
\subsubsection{Motivation}

Wireless sensor networks (WSN)  generally comprise of a group of 
spatially dispersed sensors. In a wireless sensor network, 
sensor nodes are equipped with various 
types of sensors monitoring and recording 
environmental conditions like temperature, sound, sunlight,
humidity, etc.

A given sensing task involves multiple sensors to 
achieve a certain quality-of-sensing.
Generally, an efficient task scheduling for the nodes is that nodes 
are able to perform multiple tasks simultaneously. 
For example, sensors deployed in a grove are assigned tasks to collect
sunlight, temperature and humidity data and these tasks require different 
number of  nodes with respective sensing range, rate and duration.
However, traditional sensor networks are not suitable to conduct this 
multi-task due to the limitations of computation complexity for task 
arrangement of each node.

In our {\sdn} system, we implement the multi-task application 
with the help of the central controller. The SDN controller
maintains programmable task scheduling and management
modules while sensor nodes are loaded with interfaces to
receive task control instructions.     

\subsubsection{Design}

A deployed wireless sensor network is usually assigned  with
different data collection requirement. In {\sdn}, we design and 
implement multi-task application and provide easy-to-use
interfaces to users.

When a user assigns a task to {\sdn}, the UAV controller will first check out 
the energy and storage constraints of the required sensor set, as described in Alg. \ref{Constraint}. 
If the task requirement exceeds the capacity of any selected sensor node, it will be sent to a     
task queue; Otherwise it will be put into the task buffer to conduct.


\begin{algorithm}
\caption{Sensor Constraint Detection}
\label{Constraint}
\begin{algorithmic}[1]
\STATE Input: Sensor set $N$, Task $T$;
\STATE Input: Remaining Energy for each node :$Energy(u_i)$;
\STATE Input: Remaining Storage for each node :$Energy(u_i)$;
\STATE Initialize: Energy capacity $\eta$, Storage capacity $\xi$;
\STATE Calculate the energy cost of $T$ for each sensor as : $\phi$;
\STATE Calculate the storage cost of $T$ for each sensor as :$\varphi$;
\FOR{Each node $u_i$ in $N$}
	\IF{$Energy(u_i)+\phi \geq \eta \Vert Storage(u_i)+\varphi \geq \xi$}
   	 \STATE Set $T$ to task queue;
   	 \ENDIF
\ENDFOR
\STATE Set $T$ to task buffer;
\end{algorithmic}
\end{algorithm}
We implement AI node selection and multi-task in {\sdn}, and demonstrate the code in List \ref{AI}.


\begin{lstlisting}[language={[ANSI]C},label=AI,
	caption={An example of AI selection and Muti-tasks},
	keywordstyle=\color{blue!70},
	showstringspaces=false,
	commentstyle=\color{red!50!green!80!blue!70},
	frame=single,captionpos=t,
	rulesepcolor=\color{red!20!green!20!blue!20},
	basicstyle=\ttfamily]
AI_Multitasks(taskset){
   create_scheduler();
   scheduler.create_buffer();

   for(task=taskset.head; task < taskset.len;task=task.next)
      scheduler.task_buffer_add(
         task,
         defaultset);

   scheduler.task_update();
   ...
   ...
   while((task = scheduler.task_next)!= NULL){
      data = get_collected_data();
      nodeset =
         SRSSS_selection(dataset);
      scheduler.task_buffer_update(
         task,
         nodeset);
   }
   scheduler.task_update();
}

\end{lstlisting}

\begin{table*}[htbp]
	\caption{Task Buffer}
	\label{TB}
	\centering
	\scalebox{0.9}{
	\begin{tabular}{|l|l|l|l|l|}
		\hline
		Task ID& Node set& Sensing rate & Sensing range& Sensing duration \\
		\hline
		 010& \{$u_2,u_6,...,u_{88}...u_{99}$\}& 0.2 & 4 & 2046 \\
		\hline
		...& ...& ... & ...& ... \\
		\hline
		101&  \{$u_3,u_4,...,u_{34}...u_{67}$\}& 0.2 & 5 & 4096 \\
		\hline
		...& ...& ... & ...& ... \\
		\hline
		\end{tabular}
	}
\end{table*}

Each task has a unique task ID. The task buffer maintains the task information of node set, 
sensing rate, sensing range and sensing duration. 
Every time a task is removed or updated by the user, or it runs out of its duration time,
the task queue will check to release new tasks into the task buffer.



