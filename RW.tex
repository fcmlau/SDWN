\section{Related work}

\subsection{Software Defined Wireless Sensor Networks}


There are several existing SDN approaches for sensor system, 
namely flow sensor \cite{mahmud2011exploitation}, SDWN \cite{costanzo2012software},
sensor openflow \cite{luo2012sensor}, TinySDN \cite{de2015tinysdn}, SDN-WISE \cite{galluccio2015sdn}.
However, they are all implemented and evaluated only in simulators. 

\textbf{Flow Sensor.}
The previous idea presented in paper \cite{mahmud2011exploitation} addresses 
reliability in the sensor networks through exploiting the OpenFlow technology\cite{Mckeown2008OpenFlow}. 
Coming up with the concept of flow-sensor instead of typical sensor \cite{Liu2015Thermoresistive}, 
they have successfully made it possible to achieve communications between controller, 
gateway and sensors. Besides, they proved flow sensor to be more 
reliable since data packets, control packets and the sensor nodes 
themselves can be easily monitored, regulated and routed whenever 
required. Therefore, a robust routing and uninterruptible messages 
flow of sensors are achieved. The results described in this paper 
shows flow-sensor is able to display much better performance even for large networks.

\textbf{SDWN.}
The solution of supporting the SDN approach in LR-WPANS is first presented in \cite{costanzo2012software}. 
Given the gap that advantages of SDN and the proper ways to expand it to wireless networks are not clear enough, 
the group analyzes the opportunities of SDN-related wireless network and illustrates 
the requirements should be considered to utilize SDN solution for wireless networks. 
They made a good attempt to develop the SDWN protocol stack.

\textbf{Sensor OpenFlow.}
In the paper \cite{luo2012sensor}, the group took a radical, yet backward 
and peer compatible approach to tackle the problems existing in WSN 
such as resource underutilization, counterproductive, rigidity to policy 
change and manage difficulty. They propose SD-WSN with a separation between 
data plane performing flow-based packet forwarding and control plane performing network control. 
The core part designed is Sensor OpenFlow(SOF), which gives a standard protocol for the communication 
between data and control part. Based on the whole architecture they gave, 
the underlying network becomes programmable by using SOF.

\textbf{Tiny SDN.}
TinySDN \cite{de2015tinysdn} is a TinyOS-based SDN framework aiming to 
address the problem that only single controller can be coupled to the sink. 
Besides, TinySDN is a hardware independent framework that comprises two parts: 
SDN-enabled sensor node and SDN-controller node. 
In order to test the reliability of this framework, the authors conducted some experiments on COOJA. 
Through analyzing results concerning delay and memory 
footprint, they found it is feasible in communication provided by SDN paradigm.

\textbf{SDN-WISE.}
One solution for wireless sensor network is introduced in the paper \cite{galluccio2015sdn}, 
and the authors implement the prototype of this idea. Compared to other SDN solutions for wireless network, 
this solution successfully reduces the message exchange needed between sensor nodes and controller. 
Besides, flexible APIs are provided by the authors, which makes the network much easier to program. 
Also, they implemented some experimental testbed, and found that in a power limited hardware, the response delay was still very small. 
SDN-WISE shows its good performance in different operation conditions. However, SDN-WISE only validates its performance in simulators. 


\subsection{Applications for Wireless Sensor Networks}


There are already a rich set of the applications for WSNs. 
However we study the typical applications and find that
distributed fashions for WSN cannot 
promise global optimization. %without the support of SDN.
Worse, no existing work can combine different applications to run together.
To design a ecosystem for software defined sensor networks, we are to implement applications of
routing, network diagnosis, AI sensor, sensor selection and multi-task.




\textbf{Routing.}
Low-power and lossy network consists of devices whose processing capability, 
memory and energy are constrained \cite{Winter2012}. Hence, traditional 
protocols cannot be used there. However, RPL is one of the standards for IPv6\cite{Deering1998Internet} routing in LLN. 
It builds the topology in form of a tree, performing DAO, DIO, ACK to generate a directed acyclic graph 
from one or more roots to the leaf nodes. The experiments done in the paper \cite{Tsiftes2010a} show 
that IPv6 routing with ContikiRPL is both lightweight and power-efficient.
 
\textbf{Network Diagnosis.}
Previous diagnosis algorithms share the same drawback 
that their process of gathering information is static and the reports sent to be 
analysis are not intermediate enough \cite{Ramanathan2005Sympathy,Khan2010Diagnostic}. 
Directional Diagnosis \cite{gong2015directional} is an online diagnosis approach to dynamically 
recognize the most useful information according to an engine designed in the approach. 
Besides, the diagnosis process only focuses on the problematic area, 
thus it can produce a more accuracy prediction \cite{Pavlou2015How}. 
Specifically, the whole approach consists of four main parts: 
node tracing, trace collection, inference model and incremental probing. 
The node tracing algorithms dynamically reconstruct the topical topology and derive inner dependencies. 
Based on the collected data information, inference model can guide the next probe 
by incremental probing upon belief network. The experiments done show the diagnosis 
process promises high efficiency in real time.

\textbf{AI sensor.}
The application of sensors often relays on the fields 
like sensor selection and data prediction with the help of AI algorithms. 
In order to select the subset of nodes to be active, related study gives a way to dynamically 
response to the change of sensor network based on the historical information \cite{Mo2013Dynamic}.
The new sensor node has a different time factor from the old 
node when constructing the predictive network. 
Besides, the position of nodes may be sufficiently used to reduce the weight of 
nearby nodes in a predictive network \cite{Kumar2017Edge}. 
As for the data prediction, on the one hand, the energy consumption information 
can be collected. Previous work on diagnosis, for example, only performs 
diagnosis process when sensors almost run out of energy. On the other hand, the original 
data on temperature, humidity, light intensity is a reliable resources for 
prediction in time series\cite{Raza2014Practical}. 

\textbf{Sensor Selection.}
Redundant sensor network always contains nodes that generating data 
stream with some overlapping parts \cite{Ali1995Redundant}. 
Collected information can be useful to determine which sensors are active. 
The algorithms come up with in paper\cite{li2016spatially} are called Spatially Regularized Streaming Sensor Selection(SRSSS). 
The authors not only consider the spatial information but also the historical obversion of the sensor network. 
They apply higher weights to nearby sensors and introduce a time-forgetting factor \cite{Astrom1989Adaptive} so that their approach can predict more accurately and response much faster.
	
\textbf{Multi-task.}
One single sensor node may shoulder several tasks. To save the energy consumption of the network, 
when a task is assigned, we need to choose which sensor to be active and which to be inactive \cite{Georges2011Energy}. 
Nowadays, works aiming to design an energy-efficient sensor management strategy
tends to develop their methods in three steps: choose the proper subset of nodes to be active \cite{Aghdasi2009High}, choose the subset of active nodes to assign the task and set the sampling rate of the task. Related study has found an efficient online 
algorithm considering sensor activation and task mapping to deal with dynamic events during the 
runtime of software defined sensor network system \cite{Zeng2015}.
