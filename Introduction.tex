\section{Introduction}


%logic flow


 
Software defined network (SDN) and Artificial intelligence (AI) 
are two popular topics in recent research field, 
both of which greatly improve the performance of traditional
networks.
 
On the one hand, SDN enables flexible 
programmability on traditional network system 
by using programmable data plane and centralized network controller \cite{}.
It shows remarkable advantages in good management of network control, 
less operating costs and easy deployment for new applications \cite{}.
On the other hand, AI technology promises a smarter and more effective network system
due to its high performance in predicting and optimizing network capacity,
e.g. network throughput, energy consumption, etc.

In this paper, we study integrating these two elegant techniques into 
the wireless sensor networks (WSN). Our fundamental observation is 
that conducting AI methods in WSN requires
the support of SDN. This is due to two major reasons. 
First, traditional sensor device 
has limited processing capability and cannot deal with complex computations.
Second, the sensor device is also lack of storage space 
to support complicated operations. With the help of SDN, 
the central controller can conduct complex computations 
and store huge data information instead.
  
Unfortunately, implementing a real SDN in sensor system 
is still an open problem. The existing work 
\cite{mahmud2011exploitation, costanzo2012software, luo2012sensor, de2015tinysdn, galluccio2015sdn} 
all design framework of SDN for WSN and validate them by simulations.  
Even worse, these simulational SDN works construct the central controller 
of SDN in their database station. The controller transmits control 
instructions with sensor nodes by multi-hop communications.
This causes a great deal of energy consumption, which 
makes it a paradox since sensor network ought to be an energy-efficient network.

Our insight is that, since it is unpractical to deploy SDN 
for sensor system by multi-hop control,
we refer to various equipments and techniques, 
and finally find the unmanned aerial vehicle (UAV) 
perfect to serve as the SDN controller. The flexibility
of UAV can promise one-hop communication between controller
and nodes. Furthermore, the reason why we choose UAV instead of
terrestrial mobile controller is that, the air-ground communication range
is almost 10 times than that of ground-ground communication under ZigBee protocol, which are
around 145 meters and 15 meters respectively tested by our practical experiments.
This is due to the air-ground communication suffers from less interference of the terrestrial WSN.
 

In this paper, we present {\sdn}\footnote{{\sdn} is .}, 
a real UAV based SDN sensor system.
In {\sdn}, UAV plays the role of the central controller to monitor the sensor system.
We implemented five fundamental applications: routing, network diagnosis, 
AI energy prediction, AI node selection and multi-tasks,
and provide users with easy-to-use interfaces.
Routing application allows users to update the flow table of nodes.
Network diagnosis is used to find the fault source when network exception is detected.
Multi-tasks application enables sensor to perform multiple 
tasks simultaneously. All the applications are within a ecosystem
and can be combined to achieve synergistic effect. For example,
users can execute the AI node selection application to choose
the node set for the multi-task application. 


We implemented {\sdn} with DJI M100 UAVs and CC2560 Sensor tags in real experiments, and we conducted extensive experiments for a large-scale sensor network. We deploy 130 sensors ...................
\iffalse
Adler improves localization accuracy of $20$ sensors by reducing $78.4\%$ root-mean-square error (RMSE) compared to methods by multi-hop networks or mobile vehicles. Adler achieves about $10\%$ higher package receiving ratio compared to notable mobile sink methods for gathering application. Adler reduces sensors' average energy consumption by about $80\%$ compared to multi-hop based methods. When the number of sensor nodes increases or some nodes run out of energy, Adler is more resilient and holds better performance than the state-of-the-art methods.
\fi

The main contribution of this paper is the first practically implemented SDN for sensor system, {\sdn}. 
We adopt AI tools in {\sdn} and realize five applications with easy-to-use interfaces.
We then conduct practical experiments and the evaluations validate the high performance of {\sdn}.

The paper is organized as follows. We introduce related works in the next section. 
Design of {\sdn} is presented in Section \ref{Arc}, 
and we implement five fundamental applications: routing, network diagnosis, 
AI prediction, AI node selection and multi-tasks in Section \ref{App}. 
Section \ref{Imp} gives the implement setup.
The evaluation results are provided in Section \ref{Eva} 
and we conclude the paper in Section \ref{Con}.


