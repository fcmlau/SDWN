\section{Introduction}


%logic flow

%SDN improves the traditional WSN 
 
 Software defined network (SDN) and Artificial intelligence (AI) 
 are two popular topics in recent research field, 
 both of which greatly improve the performance of traditional
 networks.
 
On the one hand, SDN enables flexible 
programmability on traditional network system 
by using programmable data plane and centralized network controller \cite{}.
It shows remarkable advantages in good management of network control, 
less operating costs and easy deployment for new applications \cite{}.
On the other hand, AI technology promises a smarter and more effective network system
due to its high performance in predicting and optimizing network capacity,
e.g. network throughput, energy consumption, etc.

In this paper, we study integrating these two elegant techniques into 
the wireless sensor networks (WSN). Our fundamental observation is 
that conducting AI methods in WSN requires
the support of SDN. This is due to two majoy reasons. 
First, traditional sensor device 
has limited processing capability and cannot deal with complex computations.
Second, the sensor device is also lack of storage space 
to support complicated operations. With the help of SDN, 
the central controller can conduct complex computations 
and store huge data information instead.
  
Unfortunately, implementing a real SDN in sensor system 
is still an open problem. The existing work \cite{} 
all design framework of SDN for WSN and validate them by simulations.  
Even worse, these simulational SDN works construct the central controller 
of SDN in their database station. The controller transmits control 
instructions with sensor nodes by multi-hop communications.
This causes a great deal of energy consumption, which 
makes it a paradox since sensor network ought to be an energy-efficient network.

Our insight is that, since it is unpractical to deploy SDN 
for sensor system by multi-hop control,
we refer to various equipments and techniques, 
and finally find the unmanned aerial vehicle (UAV) 
perfect to serve as the SDN controller.
 

In this paper, we 
UAV dd dk

In this paper, we present {\sdn}\footnote{{\sdn} means .}, a real UAV based SDN sensor system.
By utilizing Adler's application program interfaces (APIs), it is easy to evaluate any algorithm or run any application on a pure sensor network or a UAV-enable sensor system.
We implemented five fundamental applications: routing, network diagnosis, 
AI prediction, AI node selection and multi-tasks.

First, to get a sensor's three-dimension (3D) position if global position system (GPS) data is unavailable or inaccurate, we design a UAV-based localization method, which improves localization accuracy compared to methods through a multi-hop network or a mobile vehicle (as anchor nodes on the ground). Second, to gather data efficiently, we propose a UAV-based method that uses the minimum number of hexagons to cover sensors, designs flight trajectory and gathers sensors' data by UAVs. This method reduces gathering latency compared to a mobile vehicle method. Third, to update parameter settings or switch between different applications/tasks, we present a UAV-based reconfiguration method that utilizes over-the-air (OTA) programming to reconfigure sensors through one-hop communication, which reduces package loss probability compared to a multi-hop OTA programming method.

We implemented {\sdn} with DJI M100 UAVs and CC2560 Sensor tags in real experiments, and we conducted extensive simulations for large-scale sensors.
Adler improves localization accuracy of $20$ sensors by reducing $78.4\%$ root-mean-square error (RMSE) compared to methods by multi-hop networks or mobile vehicles. Adler achieves about $10\%$ higher package receiving ratio compared to notable mobile sink methods for gathering application. Adler reduces sensors' average energy consumption by about $80\%$ compared to multi-hop based methods. When the number of sensor nodes increases or some nodes run out of energy, Adler is more resilient and holds better performance than the state-of-the-art methods.


The contributions of the paper are summarized as follows:
1. We design and implement  {\sdn}, a real UAV based SDN sensor system.
2. We adopt AI tools in {\sdn} and implement five applications to smart and provide users with easy-to-use interfaces 
3. We conduct practical experiments and the evaluations 


The paper is organized as follows. We introduce related works in the next section. 
Design of {\sdn} is presented in Section \ref{Arc}, 
and we implement five fundamental applications: routing, network diagnosis, 
AI prediction, AI node selection and multi-tasks in Section \ref{App}. 
Section \ref{Imp} gives the implement setup.
The evaluation results are provided in Section \ref{Eva} 
and we conclude the paper in Section \ref{Con}.
%In our

%AI can greatly improve the sensor system

%Sensor system can improve AI system

%eco-system








\iffalse
Why and How we can implement AI ?

How we combine Ai with other applications?


AI 

AI systems have been improving, and new advances in machine intelligence are creating seamless interactions between people and digital sensor systems.

 In sensor systems, applications can be found for a variety of tasks, including selection of sensor inputs, interpreting signals, condition monitoring, fault diagnosis, machine and process control, machine design, process planning, production scheduling, and system configuring. Some examples of specific tasks undertaken by expert systems are:
* Assembly 
* Automatic programming 
* Controlling intelligent complex vehicles  
* Planning inspection 
* Predicting risk of disease 
* Selecting tools and machining strategies 
* Sequence planning 
* Controlling plant growth. 

.



The tools and methods described have minimal computation complexity and can be implemented on small assembly lines, single robots, or systems with low-capability microcontrollers. These novel approaches proposed use ambient intelligence and the mixing of different AI tools in an effort to use the best of each technology. The concepts are generically applicable across many processes.


minimum energy, data loss, reliability, robustness, etc., in place during the design and operation of wireless sensor networks

a specific set of protocols for medium access, localization and positioning, time synchronization, topology control, security and routing are identified based on the current configuration of the network, the requirements of the application and the topology of their deployment.
\fi