\section{Introduction}


%logic flow
The software defined network (SDN)~\cite{Benzekki2016Software,Sezer2013Are} paradigm
enables flexible programmability for networks 
by using programmable data plane and centralized network controller~\cite{7122247}.
It shows remarkable strengths in network control, global optimization 
(e.g., throughputs~\cite{Alicherry2005Joint}, energy consumption~\cite{Miao2010Cross}, network diagnosis~\cite{gong2015directional})
and easy deployment for new applications~\cite{Feamster2014The}.

With the support of SDN for networks, programmable artificial intelligence (AI) 
techniques~\cite{Norvig1995Artificial, Poole2010Artificial, Cockburn1996ARCHON} 
are easy to be implemented in networks \cite{mahmud2011exploitation}.
In turn, AI techniques make SDN become a smarter and more effective network system
due to their intelligence on predicting and optimizing network features~\cite{mahmud2011exploitation}
(e.g., routing~\cite{Alicherry2005Joint} and energy consumption~\cite{Miao2010Cross}).
 
 

Unfortunately, despite these advances on developing SDN and AI %Internet of Things (IoTs) \cite{Atzori2010The} 
for various networks, wireless sensor networks (WSNs)~\cite{Potdar2009Wireless}
(e.g., volcanic investigation~\cite{Wernerallen2006Deploying}
and agriculture monitoring networks~\cite{Wang2010L3SN}),
there is no real SDN implementation for WSNs. 
Existing SDN works~\cite{mahmud2011exploitation, costanzo2012software, luo2012sensor, de2015tinysdn, galluccio2015sdn} 
for WSNs are all developed in simulators.
Worse, these simulational SDN works typically construct one no-moving, centralized SDN controller 
in their simulator database~\cite{Potdar2009Wireless}. 
This often causes the sensor nodes around the  controller to incur 
excessive transmission collisions~\cite{Yang2014,Mizuyama2017Estimation} 
and energy consumption, and 
the entire system is single point of failures.
%The controller transmits control 
%instructions with sensor nodes by multi-hop communications.
%This causes a great deal of energy consumption~\cite{}, which 
%makes it a paradox since sensor network suffering from 
%ought to be an energy-efficient and reliable network.



To create a practical SDN for WSNs, 
our observation is that most computations are only suitable to run
in the SDN controllers due to two major reasons. 
First, sensor nodes 
have limited processing capability and cannot 
deal with complex computations \cite{Sharma2012Security,Heller2012The}.
Second, the sensor devices are also lack of storage space 
to support complicated operations. With the help of SDN, 
the controllers can conduct complex computations 
and store large-scale data information for WSNs.

Adler~\cite{Alder}, a recent testbed that combines WSNs
and the unmanned aerial vehicle 
(UAV)~\cite{Cambone2005Unmanned, Perry2000Unmanned, Cathcart2014Method},
shows that UAVs enable WSNs to achieve resilience, 
high-performance and energy-efficiency at the same time.
In Adler, UAVs are set to fly nearby the sensors and perform applications by only one-hop 
radio transmission.
Compared to traditional WSNs, Adler greatly reduces the network energy consumption by about $80\%$
and achieves about $42.2\%$ higher per-hop package receiving ratio.
Moreover, UAVs are capable to perform complex computations 
by their airborne computers with powerful CPUs and GPUs.
Although Adler is a traditional sensor system instead of a SDN system, 
it implies that UAVs are promising to serve as the SDN controllers.

Another advantage of utilizing UAVs as SDN controllers compared to
terrestrial mobile controllers~\cite{Somasundara2007Mobile} is that, 
the air-ground communication range
is almost 10 times than that of ground-ground communication 
under ZigBee protocol~\cite{Farahani2008ZigBee}, which are
around $150$ meters (between a UAV and a sensor) and $20$ meters(between two sensors) respectively~\cite{Alder}.
This is because air-ground communications suffer from much less interference than ground-ground.

In this paper, we present {\sdn}\footnote{In Hebrew, {\sdn} is the semantic word of angels, who fly in the sky and bring goodness to all things on earth. }, 
the first UAV based SDN system for WSNs.
In {\sdn}, a set of UAVs serve as the SDN controllers.
We design three types of SDN interfaces, including routing, multi-task and AI interfaces in {\sdn}. 
We develop five fundamental SDN applications: routing, network diagnosis, 
AI energy prediction, AI node selection and multi-task.
The routing application automatically computes flow tables for both UAVs and sensor nodes.
Therefore, sensor nodes can communicate through UAVs when they fly around,
and sensors can directly communicate with each other when UAVs are absent.
The network diagnosis application automatically finds and repair the fault routes when network exception is detected.
All these applications can greatly improve both the throughput, energy efficiency and resilience of the entire network. 
All the applications can run within a ecosystem
and can be combined synergistically. 
%For example,
%users can execute the AI node selection application to choose
%the node set for the multi-task application. 


We implement {\sdn}, using DJI M100 UAV with Intel NUC as airborne computer, 
and 100 TI CC2650 SensorTag to build our testbed.
We evaluated the five applications of  {\sdn} on this testbed, 
compared with two notable traditional routing protocols LEACH~\cite{kaur2016wsn} and RPL~\cite{winter2012rpl}.
Our evaluation shows that:
\begin{itemize}
	\item[1)] {\sdn} has high performance. {\sdn} achieves overall {\simpleTput} higher
		throughput than RPL via running only the routing application;
	\item[2)] {\sdn} has intelligent features. {\sdn} achieves absolute error of energy prediction within {\Error} 
		and obtain an {\totalLife} increment of average sensor nodes' lifetime compared with RPL;
	\item[3)] {\sdn} is resilient. As time goes by, {\sdn} maintains over
		{\pktRecvRatio} overall package received ratio while both LEACH and PRL drop to nearly {\OpktRecvRatio}
		sharply.
\end{itemize}

The main contribution of this paper is {\sdn}, the first 
practical SDN wireless sensor network system. 
{\sdn} takes the first significant step to realize global optimization in a distributed WSN environment.
Other contributions include several effective strategies and new algorithms 
in the five applications of {\sdn}. 
In contrast to all existing WSN systems, the applications in {\sdn}
can run together and benefit each other. 



The paper is organized as follows. We introduce related works in the next section. 
Design of {\sdn} is presented in Section \ref{Arc}, 
and we implement five fundamental applications: routing, network diagnosis, 
AI prediction, AI node selection and multi-task in Section \ref{App}. 
Section \ref{Imp} gives the implement setup.
The evaluation results are provided in Section \ref{Eva} 
and we conclude the paper in Section \ref{Con}.
